\documentclass[11pt]{article}

\usepackage{amsmath}           % Math functions
\usepackage{multirow}          % Tables with multirow
\usepackage{textcomp}          % For \texttildelow
\usepackage[version=3]{mhchem} % Formula subscripts using \ce{}
\usepackage[margin=1in]{geometry}


%%%%% Defining Commands to Facilitate Writing %%%%% 
% Units
\newcommand{\cm}{\ensuremath{\text{cm}^{-1}}}
\newcommand{\kcal}{\ensuremath{\text{kcal~mol}^{-1}}}
\newcommand{\dgr}{\ensuremath{^\circ}}
\newcommand{\eden}{\ensuremath{ea_{0}^{-3}}}
\newcommand{\elec}{\ensuremath{e^{-}}}
% Energy
\newcommand{\Do}{\ensuremath{D_{0}}}
\newcommand{\De}{\ensuremath{D_{\text{e}}}}
\newcommand{\dEe}{\ensuremath{\Delta E_{\text{e}}}}
\newcommand{\dEintgr}{\ensuremath{\Delta E_{\text{int,gr}}}}
\newcommand{\dEint}{\ensuremath{\Delta E_{\text{int}}}}
\newcommand{\dEcbs}{\ensuremath{\Delta E_{\text{CBS}}}}
\newcommand{\dHzero}{\ensuremath{\Delta H_{\text{0K}}}}
\newcommand{\dzpve}{\ensuremath{\Delta_\textsc{ZPVE}}}
\newcommand{\dcore}{\ensuremath{\Delta_\text{CORE}}}
\newcommand{\drel}{\ensuremath{\Delta_\text{REL}}}
\newcommand{\ddboc}{\ensuremath{\Delta_\textsc{DBOC}}}
% Other Text Shortcuts
\newcommand{\etal}{\textit{et al.}}
\newcommand{\abinitio}{\textit{ab initio}}
\newcommand{\ii}{\textit{i}}
\newcommand{\tnm}{\textnormal}
% Constants
\newcommand{\kB}{k_{\tnm{B}}}


\begin{document}

\title{Kinetics Theory}
\author{Kevin B. Moore III}
\date{}
\maketitle

\section{Canonical TST}

\begin{align}
k(T) = \kappa (T) \frac{ \kB T }{ h } \frac{m}{m^{\dagger}} 
         \frac{ Q^{\dagger} }{ Q_{\tnm{reac}} } e^{ -E / \kB T }
\end{align}

\section{Microcanonical TST}

\begin{align}
k(E) =  \frac{ N^{\dagger}(E) }{ h\rho (E) } 
\end{align}

\section{Partition Functions}




Generally use rigid-rotor, harmonic oscillator (RRHO) approximations to determine the partition functions

Generally use discrete partition function
\begin{align*}
Q = \sum_{i} g_{i} e^{ -E_{i}/\kB T }
\end{align*}

It is assumed that all of the components of $Q$ are seperable
\begin{align*}
Q = q_{vib}q_{rot}q_{trans}q_{elec}
\end{align*}
where $q_{vib}$, $q_{rot}$, $q_{trans}$, $q_{elec}$ are the partition functions describing the vibrational, rotational, translation, and electronic energy levels, respectively.


\subsection{Vibrational}

The vibrational partition function is often assumed with a harmonic oscillator models, where each vibration is an uncoupled harmonic oscillator with energy levels:
\begin{align*}
E_{vib} = h\omega
\end{align*}
where $\omega$ is the frequency of vibration. Therefore we can get the partition function for a \textit{specific} vibration where each level is equally spaced (i.e., $E_{vib,n} = nh\omega$ for the $n$-th vibrational state of mode with frequency $\omega$):
\begin{align*}
q_{vib} &= \sum_{n=1}^{\infty} e^{-E_{vib,n}/ \kB T} \\
		&= \sum_{n=1}^{\infty} e^{-nh\omega/ \kB T} \\
		&= \sum_{n=1}^{\infty} ( e^{-h\omega/ \kB T} )^{n} 
\end{align*}

This equation can be recast by recognizing that we have a convergent infinite series:
\begin{align*}
q_{vib} = \frac{1}{1-e^{-h\omega/ \kB T}} 
\end{align*}

To get the overall vibrational partition function, we note that the levels are uncoupled, and are thus seperable. Hence, the total partition function is the product of the partition functions
for each vibration:
\begin{align*}
Q_{vib_total} &= \prod_{i}^{N_{\tnm{vib}}} q_{vib,i} \\
              &= \prod_{i}^{N_{\tnm{vib}}} \frac{1}{1-e^{-h\nu/ \kB T}}   
\end{align*}

\subsection{Rotational}

\begin{align*}
E_{rot} = BJ(J+1) 
\end{align*}

\subsection{Translational}
\subsection{Electronic}

\section{Projections}

A projection operator is constructed 

\section{Tunneling}

Quantum tunneling effects placed into the $kappa{T}$ factor


\section{Instructions}







%%%%% References %%%%%
%\bibliography{kinetics-refs}

\end{document}
